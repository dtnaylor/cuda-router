
%%%%%%%%%%%%%%%%%%%%%%% file typeinst.tex %%%%%%%%%%%%%%%%%%%%%%%%%%%%%%
%
% This is the LaTeX source for the instructions to authors using
% the LaTeX document class SVMultln with class option 'lnicst'
% for contributions to the Lecture Notes of the Institute for
% Computer Sciences, Social-Informatics and
% Telecommunications Engineering series.
% www.springer.com/series/XXXX       Springer Heidelberg 2007/08/05
%
% It may be used as a template for your own input - copy it
% to a new file with a new name and use it as the basis for
% your article. It contains a few tweaked sections to demonstrate
% features of the package, though.
%
% If you have not much experiences with Springer LaTeX support,
% you should better use the special demonstration file "lnicst.tex"
% included in the LaTeX package for LNICST as template.
%
%%%%%%%%%%%%%%%%%%%%%%%%%%%%%%%%%%%%%%%%%%%%%%%%%%%%%%%%%%%%%%%%%%%%%%%%

%\documentclass{sig-alternate-10pt}
\documentclass{sig-alt-hotnets}
\usepackage{amssymb}
\setcounter{tocdepth}{3}
%\usepackage{hyperref}
\usepackage[english]{babel}
%\usepackage[pdftex]{graphicx}
\usepackage[nolist]{acronym}
\usepackage{subfigure}
\usepackage[normalem]{ulem}
\usepackage{url}
\urldef{\mailsa}\path|mukerjee@cs.cmu.edu|
\usepackage[pdfpagelabels,hypertexnames=false,breaklinks=true,bookmarksopen=true,bookmarksopenlevel=2]{hyperref}
%\renewcommand{\labelitemi}{$\bullet$}
\usepackage{cite}

%%%
%%% MISC
%%%t
\usepackage{ifthen}
\usepackage{color}
\usepackage{booktabs}  % for toprule and midrule in tables

%%%
%%% COMMENTS / TODOS
%%%
\newcommand{\showComments}{yes}

\newcommand{\note}[2]{
    \ifthenelse{\equal{\showComments}{yes}}{\textcolor{#1}{#2}}{}
}
\newcommand{\TODO}[1]{%
	\addcontentsline{tdo}{todo}{\protect{#1}}%
	\note{red}{TODO: #1}
}
\newcommand{\srini}[1]{\note{magenta}{Srini: #1}}
\newcommand{\dongsu}[1]{\note{red}{Dongsu: #1}}
\newcommand{\fahad}[1]{\note{red}{Fahad: #1}}
\newcommand{\jwb}[1]{\note{cyan}{jwb: #1}}
\newcommand{\michel}[1]{\note{blue}{Michel: #1}}
\newcommand{\aditya}[1]{\note{violet}{Aditya: #1}}
\newcommand{\ashok}[1]{\note{blue}{Ashok: #1}}
\newcommand{\dga}[1]{\note{green}{dga: #1}}
\newcommand{\hyeontaek}[1]{\note{red}{Hyeontaek: #1}}
\newcommand{\prs}[1]{\note{purple}{Peter: #1}}
\newcommand{\boyan}[1]{\note{red}{Boyan: #1}}
\newcommand{\david}[1]{\note{green}{David: #1}}
\newcommand{\matt}[1]{\note{blue}{Matt: #1}}


\makeatletter \newcommand{\listoftodos}
{\section*{Todo List} \@starttoc{tdo}}
\newcommand{\l@todo}
{\@dottedtocline{1}{0em}{2.3em}} \makeatother

\newcommand{\comment}[1]{}


%%
%% TIKZ to represent graphs and diagrams
%%

\usepackage{tikz}
\usepackage{tkz-graph}
\usetikzlibrary{shapes,arrows}

\newcommand{\entrynode}[1]{
  \SetVertexNormal[Shape      = circle,
                   FillColor  = black,
                   LineWidth  = 0pt,
                   MinSize    = 0pt]
  \Vertex[L={\tiny\,}]{#1}
  \SetVertexNormal[Shape      = circle,
                   FillColor  = white,
                   LineWidth  = 2pt]
}

\SetUpEdge[lw         = 1.5pt,
           color      = black,
           labelcolor = white,
           labeltext  = red,
           labelstyle = {sloped,draw,text=blue}]

\tikzset{node distance = 2cm}


%%%
%%% DOCUMENT
%%%

\begin{document}

%\mainmatter  % start of an individual contribution

% first the title is needed
\title{Packet Processing on the GPU}

% the name(s) of the author(s) follow(s) next
%
% NB: Chinese authors should write their first names(s) in front of
% their surnames. This ensures that the names appear correctly in
% the running heads and the author index.
%
\numberofauthors{3}
\author{
\alignauthor
Matthew Mukerjee\\
	\affaddr{Carnegie Mellon University}\\
	\email{mukerjee@cs.cmu.edu}
\alignauthor
David Naylor\\
	\affaddr{Carnegie Mellon University}\\
	\email{dnaylor@cs.cmu.edu}
	\alignauthor
Bruno Vavala\\
	\affaddr{Carnegie Mellon University}\\
	\email{bvavala@cs.cmu.edu}
} 

\maketitle


\vspace{-10pt}

\begin{abstract}

Packet processing in routers is traditionally implemented in hardware;
specialized ASICs are employed to forward packets at line rates of up to 100
Gbps. Recently, however, improving hardware and an interest in complex packet
processing has prompted the networking community to explore software routers;
though they are more flexible and easier to program, achieving forwarding rates
comparable to hardware routers is challenging.

Given the highly parallel nature of packet processing, a GPU-based router
architecture is an inviting option (and one which the community has begun to
explore). In this project, we explore the pitfalls and payoffs of implementing
a GPU-based software router. 

\end{abstract}

\section{Introduction}

There are two approaches to implementing packet processing functionality in
routers: bake the processing logic into hardware, or write it in software. The
advantage of the former is speed; today's fastest routers perform forwarding
lookups in dedicated ASICs and achieve line rates of up to 100 Gbps (the
current standard line rate for a core router is 40 Gbps)\cite{Han}. On the
other hand, the appeal of software routers is programability; software routers
can more easily support complicated packet processing functions and can be
reprogrammed with updated functionality.

Traditionally, router manufacturers have opted for hardware designs since
software routers could not come close to achieving the same forwarding rates.
Recently, however, software routers have begun gaining attention in the
networking community for two primary reasons:
\begin{enumerate}
	\item Commodity hardware has improved significantly, making it possible for
	software routers to achieve line rates. For example, PacketShader forwards
	packets at 40 Gbps. \TODO{cite}

	\item Researchers are introducing increasingly more complex packet
	processing functionality. A good example of this trend is Software Defined
	Networking \TODO{cite}, wherein each packet is classified as belonging to
	one or more \emph{flows} and is forwarded accordingly; moreover, the
	\emph{flow rules} governing forwarding behavior can be updated on the order
	of minutes or even seconds \TODO{cite}.
\end{enumerate}

Part of this new-found attention for software routers has been an exploration
of various hardware architectures that might be best suited for supporting
software-based packet processing. Since packet processing is naturally an SIMD
application, a GPU-based router is a promising candidate. The primary job of a
router is to decide, based on a packet's destination address, through which
output port the packet should be sent. Since modern GPUs contain hundreds of
cores \TODO{cite}, they could potentially perform this lookup on hundreds of
packets in parallel. The same goes for more complex packet processing
functions, like flow rule matching in a software defined network (SDN).

\TODO{Describe our goals/contributions}

The rest of this paper ... \TODO{finish}

\section{Related Work}
\label{sec:related}

We are by no means the first to explore the use of a GPU for packet processing.
We briefly summarize here the contributions of some notable projects.\\

\noindent \textbf{PacketShader \cite{Han}} implements IPv4/IPv6 forwarding,
IPsec encryption, and OpenFlow\cite{OpenFlow} flow matching on the GPU using
NVIDIA's CUDA architecture. They were the first to demonstrate the feasability
of multi-10Gbps software routers.

A second key contribution from PacketShader is a highly optimized packet I/O
engine. By allocating memory in the NIC driver for batches of packets at a time
rather than individually and by removing unneeded meta-data fields (parts of
the Linux networking stack needed by endhosts are never used in a software
router), they achieve much higher forwarding speeds, even before incorporating
the GPU. Due to time constraints, we do not make equivalent optimizations in
our NIC driver, and so we cannot directly compare our results with
PacketShader's.\\

\noindent \textbf{Gnort \cite{Vasiliadis}} ports the Snort intrusion detection
system (IDS) to the GPU (also using CUDA). Gnort uses the same basic CPU/GPU
workflow introduced by PacketShader (see \S\ref{sec:system-design}); its
primary contribution is implementing fast string pattern-matching on the GPU.\\

\noindent \textbf{Hermes \cite{Zhu}} builds on PacketShader by implementing a
CPU/GPU software router that dynamically adjusts batch size to simultaneously
optimize multiple QoS metrics (e.g., bandwidth and latency).\\

\section{Basic System Design}
\label{sec:system-design}

We use the same system design presented by PacketShader and Gnort. Pictured in
Figure~\ref{fig:system}, packets pass through our system as follows: (1) As
packets arrive at the NIC, they are copied to main memory via DMA. (2) Software
running on the CPU copies these new packets to a buffer until a sufficient
number have arrived (see below) and (3) the batch is transferred to memory on
the GPU. (4) The GPU processes the packets in parallel and fills a buffer of
results on the GPU which (5) the CPU copies back to main memory when processing
has finished. (6) Using these results, the CPU instructs the NIC(s) where to
forward each packet in the batch; (7) finally, the NIC(s) fetch(es) the packets
from main memory with another DMA and forwards them.

\begin{figure}
   \centering
   \includegraphics[scale=0.23]{figs/system_overview.pdf} 
   \caption{Basic System Design}
   \label{fig:system}
\end{figure}

\subsection{GPU Programming Issues}
\noindent \textbf{Pipelining.} As with almost any CUDA program, ours employs
pipelining for data transfers between main memory and GPU memory. While the GPU
is busy processing a batch of packets, we can utilize the CPU to copy the
results from the previous batch of packets from the GPU back to main memory and
copy the next batch of packets to the GPU (Figure~\ref{fig:pipelining}). In
this way, we never let CPU cycles go idle while the GPU is working.\\

\begin{figure}
   \centering
   \includegraphics[scale=0.25]{figs/pipelining.pdf}
   \caption{Pipelined Execution}
   \label{fig:pipelining}
\end{figure}

\noindent \textbf{Batching.} Although the GPU can process hundreds of packets
in parallel (unlike the CPU), there is, of course, a cost: non-trivial overhead
is incurred in transferring packets to the GPU and copying results back from
it. To amortize this cost, we process packets in batches... \TODO{Finish}

\subsection{Packet Processing}

A router performs one or more packet processing functions on each packet if
forwards. This includes deciding where to forward the packet (via a longest
prefix match lookup on the destination address or matching against a set of
flow rules) and might also include deciding whether or not to drop the packet
(by comparing the packet header against a set of firewall rules or comparing
the header and payload to a set of intrusion detection system rules).  In our
system we implement two packet processing functions: longest prefix match and
firewall rule matching. \TODO{Justify why?}

\subsubsection{Longest Prefix Match}
\TODO{Bruno: describe trie-based algo and DIR-24-8-BASIC}

\subsubsection{Firewall Rule Matching}

A firewall rule is defined by a set of five values (the ``5-tuple''): source
and destination IP address, source and destination port number, and protocol.
The protocol identifies who should process the packet after IP (e.g., TCP, UDP,
or ICMP). Since most traffic is either TCP or UDP \TODO{cite?}, the source and
destination port numbers are also part of the 5-tuple even though they are not
part of the network-layer header. If the protocol is ICMP, these fields can
instead be used to encode the ICMP message typce and code.

Each rule is also associated with an action (usually ACCEPT or REJECT). If a
packet matches a rule (that is, the values in the packet's 5-tuple matche those
in the rule's 5-tuple), the corresponding action is performed. Rather than
specifying a particular value, a rule might define a range of matching values
for one of the fields of its 5-tuple. A rule might also use a wildcard (*) for
one of the values if that field should not be considered when deciding if a
packet matches. For example, a corporate firewall might include the following
rule: (*, [webserver-IP], *, 80, TCP):ALLOW. This rule allows incoming traffic
from any IP address or port to access the company's webserver using TCP on port
80.

Our rule matching algorithm is incredibly simple. Rules are stored in order of
priority and a packet is checked against each one after the next in a linear
search. Though this sounds like a na\"{i}ve approach, \cite{Rovniagin} claims
that it is the approach taken by open source firewalls like \texttt{pf} and
\texttt{iptables} and likely other commercial firewalls as well. We discuss how
we generate the rules used in our tests in \S\ref{sec:eval-proc}.

\section{Evaluation}
\label{sec:eval}

\subsection{Experimental Setup}

\subsubsection{Hardware}

We ran our tests on a mid-2012 MacBook Pro with a 2.6 GHz Core i7 processor,
8GB RAM, and an NVIDIA GeForce GT 650M graphics card with 1GB memory.


\subsubsection{Router Framework}

We implemented a software router framework capable of running in one of two
modes: CPU-only or CPU/GPU. We use the router in CPU-only mode to gather a
baseline against which we can compare the performance of CPU/GPU mode. In
either mode, the framework handles gathering batches of packets, passing them
to a ``pluggable'' packet processing function, and forwarding them based on the
results of processing. 

We implemented three processing functions: one that does longest prefix match,
one that does firewall rule matching, and one that does both. For each we
implemented CPU version and a GPU version (the GPU version is a CUDA kernel
function).

\subsubsection{Packet Generation}
We use the Click Modular Router \cite{kohler2000click} to generate packets for our
software router framework to process. We modify the standard packet generation
functions in click to output UDP packets with random source and destination
addresses as well as random source and destination ports. We have click generate
these randomly addressed packets as fast as it can and have it send the packets
up to our software routing framework via a standard Berkley socket between both
locally hosted applications.

Note that this ``full-on'' kind of workload is essential to test the maximum
throughput of our router framework, but does not necessarily model a particular
kind of real-world workload. However, in the case of our system, we wish to
measure the feasibility of processing packets in software at the same speed as
specially-designed hardware ASICs. Thus our evaluation focuses on comparisons
of maximum throughput.

\subsubsection{Packet Processing}
\label{sec:eval-proc}

\noindent \textbf{Longest Prefix Match.} First of all, since CUDA provides
limited support for dynamic data structures such as a trie, we adapted the
algorithm to run on the GPU. In particular, in the initial setup of the trie,
we serialize the data structure into an array, so that it can be easily
transferred on the GPU's memory.

In order to work with a realistic dataset, we initialized the FIB with the
prefixes in the Internet belonging to actual Autonomous Systems. The list has
been retrieved from CAIDA \cite{routeviews}, which periodically stores a
simplified version of its RouteViews Collectors' BGP tables. Overall, the size
of the trie turns out to be a few tens of megabytes large. Once built, the
serialized trie is transferred to the GPU to be used during lookup.


\noindent \textbf{Firewall Rule Matching.} To evaluate the performance of our
router's firewall rule matching function, we generate a set of random firewall
rules. The 5-tuple values for our random rules are chosen according to the
distributions in \cite{Rovniagin}. For example, we use the probabilities in
Table~\ref{tab:proto-dist}, taken from \cite{Rovniagin}, to pick a new random
rule's protocol. Similar statistics were used to pick source/destination
address/port.

\begin{table}[htbp]
   \centering
   \begin{tabular}{ l l } 
      \toprule
      \textbf{Protocol}  & \textbf{Prevelance in Rules} \\
      \midrule
	  TCP & 75\% \\
      UDP & 14\% \\
	  ICMP & 4\% \\
	  * & 6\% \\
	  Other & 1\% \\
      \bottomrule
   \end{tabular}
   \caption{Protocol distribution in firewall rules}
   \label{tab:proto-dist}
\end{table}

\subsection{Evaluation Metrics}
\label{sec:metrics}

We use three different metrics to evaluate our router's performance. The results
in \S\ref{sec:results} present these values averaged over 30 seconds of router
operation.

\medskip \noindent \textbf{Bandwidth.} No router performance evaluation would
be complete without considering bandwidth, and so we measure the number of 64
byte packets forwarded by our router per second, from which we calculate
bandwidth in gigabits per second. For comparison, the norm for core routers is
about 40 Gbps with high-end routers currently maxing out around 100 Gbps.

\medskip \noindent \textbf{Latency.} Bandwidth only tells part of the story,
however; the delay a single packet experiences at a router is important as well
(for some applications, it is more important than bandwidth). Since some
optimizations aimed at increasing our router's bandwidth increase latency as
well (such as increasing batch sizes), measuring latency is important.

We measure both the minimum and maximum latencies of our router. The maximum
latency is the time from the arrivial of the first packet of a batch to the
time it is forwarded; the minimum latency is the same but for the last packet
of a batch.

\medskip \noindent \textbf{Processing Time.} We also consider a microbenchmark:
the time spent in the packet processing function itself (i.e., doing the longest
prefix match lookup or matching against firewall rules). This is largely a
sanity check; we expect to see the GPU perform much better here, though actual
performance (measured by bandwidth and latency) depends on many other factors
(like the time spent copying data to/from the GPU).


\subsection{Results}
\label{sec:results}

Our router went through four iterations, each one introducing optimizations
based on what we learned from the results of the last. We therefore present our
results in four stages, guiding the reader through our design process.

\subsubsection{Iteration One}

We began by na\"{i}vely implementing the design presented in
\S\ref{sec:system-design}; at this point, we made no optimizations --- the goal
was building a functioning GPU-based router.

Not surprisingly, it performed underwhelmingly. The GPU version of our router
achieved roughly 80\% of the bandwidth the CPU version did
(Figure~\ref{fig:iter1-bw}) and its (max and min) latencies were 1.3X longer
(Figure~\ref{fig:iter2-lat}). The processing time is not to blame; as expected,
the GPU performs the actual packet processing much faster than the CPU
(Figure~\ref{fig:iter1-proc} --- the GPU processing time is so small it hugs
the X axis). A quick examination of the time spent performing different
functions (Figure~\ref{fig:iter1-gpu-breakdown}) explains where the GPU router
is losing ground: although it spends less time processing, it has to copy
packets to the GPU for processing and copy the results back, tasks the CPU
version doesn't need to worry about (Figure~\ref{fig:iter1-cpu-breakdown}).

\begin{figure}
    \centering
    \subfigure[Bandwidth vs. Batch Size]{\includegraphics[height=3.5cm]{figs/batch_size_both_par_bw.pdf}\label{fig:iter1-bw}}
	\medskip

    \subfigure[Latency vs. Batch Size]{\includegraphics[height=3.5cm]{figs/batch_size_both_par_lat.pdf}\label{fig:iter1-lat}}

	\medskip
    
	\subfigure[Processing Time vs. Batch Size]{\includegraphics[height=3.5cm]{figs/batch_size_both_par_proc.pdf}\label{fig:iter1-proc}}
	\medskip
	
	\subfigure[Breakdown of Relative GPU Time]{\includegraphics[height=3.5cm]{figs/batch_size_both_par_gpu_p.pdf}\label{fig:iter1-gpu-breakdown}}
	
	\medskip
	\subfigure[Breakdown of Relative CPU Time]{\includegraphics[height=3.5cm]{figs/batch_size_both_par_cpu_p.pdf}\label{fig:iter1-cpu-breakdown}}

    \caption{Iteration 1 Results}
	\label{fig:iter1}
\end{figure}


\subsubsection{Iteration Two}

Since both of our packet processing functions operate only on the data carried
by the packet header, we can reduce the time spent copying data to the GPU by
copying only the packet headers. Unfortunately, the results do not contain
unnecessary information, and cannot easily be condensed (this is not completely
true --- it is probably possible to compress the results, though we do not
explore this here).

Figure~\ref{fig:iter2} shows the performance of the second iteration of our
router. Reducing the number of bytes copied to the GPU has closed the gap
between the CPU-only and the CPU/GPU routers in terms of bandwidth and latency,
but the CPU/GPU router still doesn't perform any better. Even though we have
all but eliminated copy time to GPU, Figure~\ref{fig:iter2-gpu-breakdown}
suggests that we should try to cut down copy time from the GPU as well.

\begin{figure}
    \centering
    \subfigure[Bandwidth vs. Batch Size]{\includegraphics[height=4cm]{figs/batch_size_header_both_par_bw.pdf}\label{fig:iter2-bw}}

	\medskip
    \subfigure[Latency vs. Batch Size]{\includegraphics[height=4cm]{figs/batch_size_header_both_par_lat.pdf}\label{fig:iter2-lat}}

   	\medskip
	\subfigure[Breakdown of Relative GPU Time]{\includegraphics[height=4cm]{figs/batch_size_header_both_par_gpu_p.pdf}\label{fig:iter2-gpu-breakdown}}

    \caption{Iteration 2 Results}
	\label{fig:iter2}
\end{figure}


\subsubsection{Iteration Three}

Unfortunately, there is no unnecessary data being copied back from the GPU as
there was being copied to it; we must find another way to reduce the
copy-from-GPU overhead. Instead, we modify our router's workflow to take
advantage of mapped memory (\S\ref{sec:gpu-issues}).

This gives the CPU/GPU router a tiny edge over the CPU-only router
(Figure~\ref{fig:iter3}), but the gains are small (as Amdahl's Law would
suggest --- the copy-from-device time we eliminated was a small portion of the
total time in Figure~\ref{fig:iter2-gpu-breakdown}).


\begin{figure}
    \centering
    \subfigure[Bandwidth vs. Batch Size]{\includegraphics[height=4cm]{figs/batch_size_header_pinned_both_par_bw.pdf}\label{fig:iter3-bw}}

	\medskip
    \subfigure[Latency vs. Batch Size]{\includegraphics[height=4cm]{figs/batch_size_header_pinned_both_par_lat.pdf}\label{fig:iter3-lat}}

   	\medskip
	\subfigure[Breakdown of Relative GPU Time]{\includegraphics[height=4cm]{figs/batch_size_header_pinned_both_par_gpu_p.pdf}\label{fig:iter3-gpu-breakdown}}

    \caption{Iteration 3 Results}
	\label{fig:iter3}
\end{figure}


\subsubsection{Iteration Four}

Having eliminated the overhead of copying data to and from the GPU, the third
iteration of our CPU/GPU router only displays meager gains over the CPU-only
version. We turn to a breakdown by function of the absolute runtime of each
(Figure~\ref{fig:iter3-breakdown}) to understand why.
\begin{figure*}
    \centering
    \subfigure[Breakdown of Absolute GPU Time]{\includegraphics[height=3.7cm]{figs/batch_size_header_pinned_both_par_gpu.pdf}\label{fig:iter3-gpu-breakdown}}
	\quad
    \subfigure[Breakdown of Absolute CPU Time]{\includegraphics[height=3.7cm]{figs/batch_size_header_pinned_both_par_cpu.pdf}\label{fig:iter3-cpu-breakdown}}

    \caption{Iteration 3 Breakdown}
	\label{fig:iter3-breakdown}
\end{figure*}

The CPU-only router clearly spends more time in the processing function;
however, both spend nearly all their time performing packet I/O (that is,
receiving packets from Click and forwarding them after processing), so the
difference is processing times has little effect. This suggests that our
Click-based packet generator is the bottleneck. (Indeed, PacketShader spent a
great amount of time developing highly optimized drivers for their NIC
\cite{Han}.) To test this hypothesis, we implemented a version of our packet
generator that pre-generates a buffer of random packets and returns this buffer
immediately when queried for a new batch; similarly, when results are returned
so that packets may be forwarded, the call returns immediately rather than
waiting for the whole batch to be sent.

Sure enough, the CPU/GPU router now operates at 3X the bandwidth of the
CPU-only version and with $1/5$th the latency (Figure~\ref{fig:iter4}). Of
course, the bandwidth achieved by both is completely unrealistic (instantaneous
packet I/O is impossible), but these results indicate that with optimized
packet I/O drivers like PacketShader's, our CPU/GPU router would indeed
outperform our CPU-only one.

\begin{figure*}
    \centering
    \subfigure[Bandwidth vs. Batch Size]{\includegraphics[height=4.5cm]{figs/batch_size_header_pinned_immediate_both_par_bw.pdf}\label{fig:iter4-bw}}
	\quad
    \subfigure[Latency vs. Batch Size]{\includegraphics[height=4.5cm]{figs/batch_size_header_pinned_immediate_both_par_lat.pdf}\label{fig:iter4-lat}}

    \caption{Iteration 4 Results}
	\label{fig:iter4}
\end{figure*}

\section{Discussion and Future Work}

\noindent \textbf{Multithreaded CPU Processing.} The comparison of our CPU/GPU
router to the CPU-only baseline is not completely fair. Our CPU-only router
operates in a single thread, yielding misleadingly low performance. Any
CPU-based software router in the real world would certainly spread packet
processing across multiple cores, and we would be surprised if any core
software router were run on a machine with fewer than 16 cores.

A simple extension of our project would be to parallelize our CPU-only packet
processing functions with OpenMP to provide more realistic baseline
measurements.

\medskip \noindent \textbf{Harder Processing Functions.} Even in the final
iteration of our router, the CPU/GPU version only achieves slightly more than
three times the bandwidth of the CPU-only version. Though this is by no means
an improvement to scoff at, the speedup strikes us as being a tad low. We
suspect the cause is that our packet processing functions are not taxing
enough; the harder the processing function, the more benefit we should see from
the massively parallel GPU. This suggests that GPU-based software routers might
be best suited for complex packet processing like IDS filtering (which requires
pattern matching against packet payloads) and IPsec processing (which requires
expensive cryptographic operations).

\medskip \noindent \textbf{Faster Packet I/O.} \TODO{Matt: Briefly talk about how we really need better packet io, like PacketShader's drivers? Mention that Figure \ref{fig:iter4} supports this.}

\medskip \noindent \textbf{Streams.} \TODO{Bruno: Briefly describe CUDA's built-in streams funcitonality}

\medskip \noindent \textbf{Integrated Graphics Processors.} \TODO{Matt: Briefly describe the stuff Srini suggested}

\section{Conclusion}
\label{sec:concl}

\TODO{general conclusion stuff}



\newpage
{\small
\bibliographystyle{abbrv}
\bibliography{refs}
}

\appendix
\section{Distribution of Credit}
\begin{center}
   \begin{tabular}{ l l } 
      \toprule
      \textbf{Group Member}  & \textbf{Portion of Credit} \\
      \midrule
	  Matt & 33.3\% \\
      David & 33.4\% \\
	  Bruno & 33.3\% \\
      \bottomrule
   \end{tabular}
\end{center}


\end{document}
