\section{Related Work}
\label{sec:related}

We are by no means the first to explore the use of a GPU for packet processing.
We briefly summarize here the contributions of some notable projects.\\

\noindent \textbf{PacketShader \cite{Han}} implements IPv4/IPv6 forwarding,
IPsec encryption, and OpenFlow\cite{OpenFlow} flow matching on the GPU using
NVIDIA's CUDA architecture. They were the first to demonstrate the feasability
of multi-10Gbps software routers.

A second key contribution from PacketShader is a highly optimized packet I/O
engine. By allocating memory in the NIC driver for batches of packets at a time
rather than individually and by removing unneeded meta-data fields (parts of
the Linux networking stack needed by endhosts are never used in a software
router), they achieve much higher forwarding speeds, even before incorporating
the GPU. Due to time constraints, we do not make equivalent optimizations in
our NIC driver, and so we cannot directly compare our results with
PacketShader's.\\

\noindent \textbf{Gnort \cite{Vasiliadis}} ports the Snort intrusion detection
system (IDS) to the GPU (also using CUDA). Gnort uses the same basic CPU/GPU
workflow introduced by PacketShader (see \S\ref{sec:system-design}); its
primary contribution is implementing fast string pattern-matching on the GPU.\\

\noindent \textbf{Hermes \cite{Zhu}} builds on PacketShader by implementing a
CPU/GPU software router that dynamically adjusts batch size to simultaneously
optimize multiple QoS metrics (e.g., bandwidth and latency).\\
