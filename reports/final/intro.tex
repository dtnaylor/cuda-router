\section{Introduction}

There are two approaches to implementing packet processing functionality in
routers: bake the processing logic into hardware, or write it in software. The
advantage of the former is speed; today's fastest routers perform forwarding
lookups in dedicated ASICs and achieve line rates of up to 100 Gbps (the
current standard line rate for a core router is 40 Gbps)\cite{Han}. On the
other hand, the appeal of software routers is programability; software routers
can more easily support complicated packet processing functions and can be
reprogrammed with updated functionality.

Traditionally, router manufacturers have opted for hardware designs since
software routers could not come close to achieving the same forwarding rates.
Recently, however, software routers have begun gaining attention in the
networking community for two primary reasons:
\begin{enumerate}
	\item Commodity hardware has improved significantly, making it possible for
	software routers to achieve line rates. For example, PacketShader forwards
	packets at 40 Gbps \cite{Han}.

	\item Researchers are introducing increasingly more complex packet
	processing functionality. A good example of this trend is Software Defined
	Networking \cite{OpenFlow}, wherein each packet is classified as belonging to
	one or more \emph{flows} and is forwarded accordingly; moreover, the
	\emph{flow rules} governing forwarding behavior can be updated on the order
	of minutes \cite{OpenFlow} or even seconds \cite{Jafarian}.
\end{enumerate}

Part of this new-found attention for software routers has been an exploration
of various hardware architectures that might be best suited for supporting
software-based packet processing. Since packet processing is naturally an SIMD
application, a GPU-based router is a promising candidate. The primary job of a
router is to decide, based on a packet's destination address, through which
output port the packet should be sent. Since modern GPUs contain hundreds of
cores \cite{Ryoo}, they could potentially perform this lookup on hundreds of
packets in parallel. The same goes for more complex packet processing
functions, like flow rule matching in a software defined network (SDN).

\TODO{Describe our goals/contributions}

The rest of this paper ... \TODO{finish}
