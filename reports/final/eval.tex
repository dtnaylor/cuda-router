\section{Evaluation}
\label{sec:eval}

\subsection{Experimental Setup}

\subsubsection{Hardware}

We ran our tests on a mid-2012 MacBook Pro with a 2.6 GHz Core i7 processor,
8GB RAM, and an NVIDIA GeForce GT 650M graphics card with 1GB memory.

\subsubsection{Packet Generation}
\TODO{Matt: describe using click generator for ``full-on'' tests and packet trace for ``real-world'' tests}

\subsubsection{Packet Processing}
\label{sec:eval-proc}

\noindent \textbf{Longest Prefix Match.} Describe the FIB we use...\\
\TODO{Bruno: describe using RouteViews? FIB...}

\noindent \textbf{Firewall Rule Matching.} To evaluate the performance of our
router's firewall rule matching function, we generate a set of random firewall
rules. The 5-tuple values for our random rules are chosen according to the
distributions in \cite{Rovniagin}. For example, we use the probabilities in
Table~\ref{tab:proto-dist}, taken from \cite{Rovniagin}, to pick a new rule's
protocol. Similar statistics were used to pick source/destination address/port.

\begin{table}[htbp]
   \centering
   \begin{tabular}{ l l } 
      \toprule
      \textbf{Protocol}  & \textbf{Prevelance in Rules} \\
      \midrule
	  TCP & 75\% \\
      UDP & 14\% \\
	  ICMP & 4\% \\
	  * & 6\% \\
	  Other & 1\% \\
      \bottomrule
   \end{tabular}
   \caption{Protocol distribution in firewall rules}
   \label{tab:proto-dist}
\end{table}

\subsection{Evaluation Metrics}
\label{sec:metrics}

We use three different metrics to evalute our router's performance. The results
presented in \S\ref{sec:results} present these values averaged over 30 seconds
of router operation.

\noindent \textbf{Bandwidth.} \\

\noindent \textbf{Latency.} \\

\noindent \textbf{Processing Time.}


\subsection{Results}
\label{sec:results}

\TODO{Finish... I'm about to generate some new plots...}
